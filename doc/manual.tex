\documentclass[a4paper,11pt]{article}

\usepackage[utf8]{inputenc}
\usepackage[czech]{babel}
\usepackage[left=2cm,top=3cm,text={17cm,24cm}]{geometry}
\usepackage{graphicx}
\usepackage{listings}
\usepackage{url}

\title{IPK - Počítačové komunikace a sítě\\
{\bf\large Varianta 2: Klient-server pro jednoduchý přenos souborů (Veselý)}}

\author{Vysoké učení technické v Brně}

\date{\url{https://github.com/ldrahnik/ipk_1_project_2}}

\date{Lukáš Drahník (xdrahn00), 10.03.2018}

\begin{document}

{\let\newpage\relax\maketitle}

\newpage

\section*{Obsah}
\begin{itemize}
  \item Úvod
  \item Návrh programu
  \item Implementace
  \item Použití aplikace
  \item Závěr
\end{itemize}

\newpage

\section*{Úvod}
\begin{itemize}
  \item Dokumentace má za účel vysvětlit jak došlo k návrhu programu co se především týče k zadání, jeho implementace a způsob použití.
\end{itemize}

\section*{Návrh programu}
\begin{itemize}
  \item Spojení inicializuje klient (musí znát hostname a port, na kterém server naslouchá) a po
úspěšném navázání TCP spojení odešle hlavičku s požadavkem, délkou jména souboru a samotným jménem souboru. Server na
ní reaguje odpovědí se stavem (zda je vše OK, nebo soubor neexistuje, nastala nějaká
chyba při otevírání, zamykání souboru). Po přijetí odpovědi klient buď přijímá data nebo je odesílá nebo se ukončí s příslušnou
chybovou hláškou. Vysílající uzavře po odeslání celého souboru socket. Po dokončení přenosu případně server dle údaju poskytnutým v hlavičce při módu write zkontroluje,
zda přijatý soubor je stejně velký jako avizovaný na začátku přenosu.
\end{itemize}

\section*{Implementace}
\begin{itemize}
  \item Hlavička je tvořena 1 B určujícím mode (read nebo write), 4 B (sizeof(int), obsahující délku jména souboru a poté samotný název souboru. Při mode read poté write poté ještě následuje informace o délce souboru typu sizeof(long).
Tuto hlavičku přijímá server postupně, zjistí požadovanou operaci a kolik B má název souboru. Pomocí této délky
pak přijme zbytek hlavičky a zkusí otevřít soubor. Nakonec odpovídá 1 B odpovědí se
stavovým kódem. Pokud vše proběhlo bez chyby, začne přijímat/odesílat data.
Po přenesení celého souboru odesílající uzavře socket a spojení je ukončeno. Při mode write server ví, alespoň dle velikosti souboru,
zda přijmul stejně velký a může si to po dokončení přenosu zkontrolovat.
\end{itemize}

\section*{Použití aplikace}
\begin{itemize}

  \item
  \lstset{language=Bash}
  \begin{lstlisting}[frame=single,breaklines]
    ./ipk-client -h eva.fit.vutbr.cz -p 55555 -r myfile.xml
  \end{lstlisting}

  \item
  \lstset{language=Bash}
  \begin{lstlisting}[frame=single,breaklines]
    ./ipk-server -p port
  \end{lstlisting}

\end{itemize}

\section*{Závěr}
\begin{itemize}
  \item Projekt pomohl pochopit úskalí návrhu protokolu, složité rozhodování týkající se multiplatformosti, jednoduchosti, použitelnost a dalších aspektů.
\end{itemize}

\nocite{*}

%% BIBLIOGRAPHY
\bibliography{local}
\bibliographystyle{plain}

\newpage
\thispagestyle{empty}

\end{document}
%% END OF FILE
